\chapter*{Introduction}

In order to properly implement the design decisions made for the PLeTs v4 project, it is necessary that all professional involved be aware of certain mechanisms, both theoretical and practical, that will be used in its design and development. This document is aimed to serve both as a basis for the study and explanation of these techniques and as a guide for the appropriate representation and implementation of them in PLeTs v4.

In Chapter \ref{ch:staticfinal} we will present the notion of using compilation directives for conditional compilation. First a basic theory of the concept is presented, based largely on the author's experience. The mechanism of using Static Final variables in Java is presented, along with its advantages and limitations. Then, the mechanisms of using Java annotations are evaluated. Finally, we analyse the need and properties of using an external preprocessor software, along with its perceived advantages.

In Chapter \ref{ch:patterns} we will present the basics of the software design patterns and architectural patterns used in PLeTs v4, with a focus on explaining their goals and implementation. First the Variant Factory design pattern will be presented \cite{LASER:2015} alongside the data structure catalogue system implemented in PLeTs v3. We then briefly expose the basic concepts of Software Connectors \cite{TAYLOR:2009}, with a particular emphasis on Linkage and Arbitrator connectors. With a basis on these concepts, we then proceed to present the architectural pattern proposed for the composition of products in PLeTs v4, which is largely based on the Distributed Objects architectural style (which, as far as we could find, was first used in CORBA \cite{CORBA:2012}).

Chapter \ref{ch:modules} is meant to serve as a guide to the components designed within PLeTs v4 and their specifications. This chapter is focused on the architectural characteristics of these components, and is not meant to serve as a guide to the software testing methods applied in PLeTs.
