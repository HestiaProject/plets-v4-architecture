\chapter{PLeTs v4 Modules}
\label{ch:modules}

\textcolor{red}{Brief introduction of what PLeTs is, presentation of the Feature Model, explanation of the purpose of the project from the point of view of the Testing domain. 2-4 paragraphs should be enough.}

%----------------------------------------------------------

\section{Variability Point 1}

\textcolor{red}{Description of the variability point, what its role is in the product line, what are the typical input and output artifacts. 1-3 paragraphs per variability point should be enough.}

%----------------------------------------------------------

\subsection{Variant 1.A}

\textcolor{red}{Very brief description of the variant (1 paragraph) and definition of the interface in the following format or similar:}

\begin{listing}
In:
\\  - Artifact 1 (described in \ref{})
\\  - Artifact 2 (described in \ref{})
\\Out:
\\  - Artifact 3 (described in \ref{})
\\Inout:
\\  - Artifact 4 (described in \ref{})
\\Pre-Condition:
\\  - Constraint a
\\  - Constraint b
\\Post-Condition:
\\  - Constraint c
\caption{Variant 1.A Interface}
\end{listing}

\textcolor{red}{Artifacts will generally be files or physical data structures. Inout artifacts represent artifacts that are both input and output. Constraints will typically be a brief description of the process executed by the variant, and rules as to what may or may not be modified (in the artifacts) over the course of its operation. Not all Variants will have Constraints, and constraints are typically applied to inout artifacts.}

%----------------------------------------------------------

\subsection{Variant 1.B}

%----------------------------------------------------------

\section{Variability Point 2}

%----------------------------------------------------------

\subsection{Variant 2.A}

%etc.
