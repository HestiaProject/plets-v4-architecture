\documentclass{report}
\usepackage[utf8]{inputenc}
\usepackage{graphicx}
\usepackage{listings}
\usepackage{minted}
\usepackage{syntax}
\usepackage[acronym]{glossaries}

\title{PLeTs v4 Architectural Manual}
\author{Marcelo Schmitt Laser}
\date{April 2017}

%-----------------------------------------------

\makeglossaries

\newacronym{bnf}{BNF}{Backus-Naur Form}
\newacronym{corba}{CORBA}{Common Object Request Broker Architecture}
\newacronym{cbse}{CBSE}{Component-Based Software Engineering}
\newacronym{spl}{SPL}{Software Product Line}
\newacronym{sple}{SPLE}{Software Product Line Engineering}
\newacronym{sei}{SEI}{Software Engineering Institute}
\newglossaryentry{bindtime}
{
	name={binding time},
	description={The moment during which the properties of a software artifact are configured. Typically, these may be: compile-time, where the configuration is handled during the compilation process; link-time, where the configuration is handled at the moment of dependency solution and access to external libraries; load-time, where the configuration is handled the moment the software product is instanced (loaded), and; run-time, where the configuration is handled during the software's execution}
}
\newglossaryentry{grammar}
{
	name={grammar},
	description={The set of rules that define the structure of all possible strings within a language. For any one string (regardless of size) to belong to a particular language, it must match that language's grammar. Unless stated otherwise, this document uses the terms \emph{grammar} and \emph{context-free grammar} interchangeably. More information on context-free grammars and grammars in general can be found in \cite{HOPCROFT:2000}}
}
\newglossaryentry{regex}
{
	name={regular expression},
	description={A sequence of characters that represents a string pattern. A regular expression has several uses, among which the ones we are most interested in are searching a stream of text for matching strings and validating a string against a pre-determined pattern. More information on regular expressions can be found in \cite{HOPCROFT:2000}}
}
\newglossaryentry{variability}
{
	name={variability},
	description={From SEI \cite{BACHMANN:2005}: ``Variability is the ability of a system, an asset, or a development environment to support the production of a set of artifacts that differ from each other in a preplanned fashion.''}
}
\newglossaryentry{variant}
{
	name={variant},
	description={From SEI \cite{BACHMANN:2005}: ``The realization of a variable part, meaning the result of exercising the variation mechanism(s), is called a variant.''}
}

%-----------------------------------------------

\begin{document}

%\maketitle

\begin{titlepage}
	\centering
	\par\vspace{3cm}
	{\scshape\LARGE Universidade Federal do Pampa \par}
	\vspace{4cm}
	{\huge\bfseries Introduction to Variability Mechanisms in Java\par}
	\vspace{2cm}
	{\Large\itshape Marcelo Schmitt Laser\par}

	\vfill

% Bottom of the page
	{\large \today\par}
\end{titlepage}

\chapter*{Introduction}

In order to properly implement the design decisions made for the PLeTs v4 project, it is necessary that all professional involved be aware of certain mechanisms, both theoretical and practical, that will be used in its design and development. This document is aimed to serve both as a basis for the study and explanation of these techniques and as a guide for the appropriate representation and implementation of them in PLeTs v4.

In Chapter \ref{ch:staticfinal} we will present the notion of using compilation directives for conditional compilation. First a basic theory of the concept is presented, based largely on the author's experience. The mechanism of using Static Final variables in Java is presented, along with its advantages and limitations. Then, the mechanisms of using Java annotations are evaluated. Finally, we analyse the need and properties of using an external preprocessor software, along with its perceived advantages.

In Chapter \ref{ch:patterns} we will present the basics of the software design patterns and architectural patterns used in PLeTs v4, with a focus on explaining their goals and implementation. First the Variant Factory design pattern will be presented \cite{LASER:2015} alongside the data structure catalogue system implemented in PLeTs v3. We then briefly expose the basic concepts of Software Connectors \cite{TAYLOR:2009}, with a particular emphasis on Linkage and Arbitrator connectors. With a basis on these concepts, we then proceed to present the architectural pattern proposed for the composition of products in PLeTs v4, which is largely based on the Distributed Objects architectural style (which, as far as we could find, was first used in CORBA \cite{CORBA:2012}).

Chapter \ref{ch:modules} is meant to serve as a guide to the components designed within PLeTs v4 and their specifications. This chapter is focused on the architectural characteristics of these components, and is not meant to serve as a guide to the software testing methods applied in PLeTs.


\printglossary[type=\acronymtype]
\printglossary

\tableofcontents

\chapter{Compilation Directives}
\label{ch:staticfinal}

Compilation directives are well-known in C++, often identified as "\#ifdef". While C\# made a similar mechanism available to us (\#if), Java is lacking in an in-built preprocessor, and therefore has no mechanism with the exact same purpose as the C++ compilation directives. This chapter presents a short introduction to compilers and the concept of compilation directives and preprocessors, as well as the reasons why these are used in PLeTs and how to use Static Final variables in Java to reach a similar result.

%----------------------------------------------------------

\section{Compilers}

Most are unaware of the exact way through which compilers work, but a basic understanding of their mechanisms is necessary to fully comprehend the uses of compilation directives. Compilers can generally be divided in three or four smaller tools that, together, execute the entire compilation process. These are the Lexical Analyser, the Syntactic Analyser, the Semantic Analyser, and, optionally, the Code Generator. Other features can be included in compilers, such as code optimizers, but these are not covered here.

%----------------------------------------------------------

\subsection{Lexical Analyser}

The Lexical Analyser is the part of a compiler that is responsible for tokenizing the input text. What this means is that the text, generally code, that is sent as input to the compiler is fully read and divided into its basic units, such as "identifiers", "reserved words", etc. This is important for the compilation process as the following steps (Syntactic and Semantic analysis) are not entirely concerned with the exact text included in the input, but rather with its structure. This tokenization process is generally executed in a single pass, and outputs an ordered list of tokens.

We can use the typical "Hello World" Java program as an example of Lexical Analysis, based on the code presented in Listing \ref{lst:helloworldlex}, taken from \cite{SEDGEWICK:2011}.

\begin{listing}
\begin{minted}[frame=single,
               framesep=3mm,
               linenos=true,
               xleftmargin=21pt,
               breaklines=true,
               tabsize=4]{java}
public class HelloWorld {
  public static void main(String[] args) {
      // Prints "Hello, World" to the terminal window.
      System.out.println("Hello, World");
     }
}
\end{minted}
\caption{HelloWorld.java from \cite{SEDGEWICK:2011}} \label{lst:helloworldlex}
\end{listing}

\chapter{Architectural and Design Patterns in PLeTs v4}
\label{ch:patterns}

While Chapter \ref{ch:staticfinal} focused on the specific code mechanisms used to implement variability management in Java, this chapter approaches the problem from a higher-level perspective, analysing patterns that may be applied to solve common problems that arise during the course of design and development. While the topics of design patterns \cite{GAMMA:1995} and architectural patterns \cite{TAYLOR:2009} are very deeply studied in several excellent works, this document seeks to apply a specific few to solving the problems that were identified in PLeTs in the past (v1-3) and which are expected to arise over the course of PLeTs v4.

The differences between architectural and design patterns are substantial and very important (see \cite{TAYLOR:2009}), but for the purposes of this document they will be treated as virtually the same. Many of the problems raised in this chapter in the current version of this document are still open problems in the design and architecture of PLeTs v4, and therefore any input provided is welcome.

%----------------------------------------------------------

\section*{How to read this chapter}

This chapter is written to serve as both an introduction to the patterns used in PLeTs v4 and as a reference guide for use over the course of the project. Any information regarding the situations in which to apply a pattern or form of its implementation should first be sought in this document.

Section \ref{sc:factory} briefly presents the Factory design pattern, based on the original description by Gamma et al. \cite{GAMMA:1995}, as well as the more specific application of it in PLeTs, called the Variant Factory \cite{LASER:2015}.

Section \ref{sc:connectors} \emph{very} briefly presents the notions of software \gls{connector}s, based on the description by Taylor et al. \cite{TAYLOR:2009}, with an emphasis on the linkage- and arbitrator-type \gls{connector}s.

Section \ref{sc:builder} follows up on these concepts with the idea of the architectural pattern behind the dynamic composition and communication of modules within products that is the cornerstone of PLeTs v4. We explore two architectural styles, Distributed Objects (best known for its flagship framework, \acrshort{corba} \cite{CORBA:2012}) and C2 (described in detail in \cite{TAYLOR:2009}), and how ideas from each of them were used to conceive the ``Builder'' \textcolor{red}{(name pending)} Architectural Pattern for PLeTs v4.

%----------------------------------------------------------

\section{Factory Pattern}
\label{sc:factory}

%----------------------------------------------------------

\subsection{Variant Factory}

%----------------------------------------------------------

\section{Software Connectors}
\label{sc:connectors}

%----------------------------------------------------------

\subsection{Linkage Connectors}

%----------------------------------------------------------

\subsection{Arbitrator Connectors}

%----------------------------------------------------------

\section{PLeTs v4 Architectural Pattern - ``Builder''}
\label{sc:builder}

%----------------------------------------------------------

\subsection{Distributed Objects Architectural Style}

%----------------------------------------------------------

\subsection{C2 Architectural Style}

\chapter{PLeTs v4 Modules}
\label{ch:modules}

%----------------------------------------------------------

\section{Input Processing}

\\The process of Input Processing is bla bla bla, it is used bla bla bla. In our work we have mainly used bla bla bla, through techniques of bla bla bla.

%----------------------------------------------------------

\subsection{XML Reader}

%----------------------------------------------------------

\section{Test Sequence Generation}

\\Test Sequence Generation modules are in charge of bla bla bla, they use techniques such as bla bla bla.

%----------------------------------------------------------

\subsection{Depth-First Search}

%----------------------------------------------------------

\subsection{Breadth-First Search}

%----------------------------------------------------------

\section{Test Case Generator}

%----------------------------------------------------------

\subsection{Performance Test Case Generator}

%----------------------------------------------------------

\section{Test Script Generation}

%----------------------------------------------------------

\subsection{JMeter Script Generator}

%----------------------------------------------------------

\subsection{Oracle Application Testing Suite (OATS) Script Generator}

%----------------------------------------------------------

\section{Test Executor}

%----------------------------------------------------------

\subsection{JMeter Test Executor}

%----------------------------------------------------------

\subsection{Oracle Application Testing Suite (OATS) Test Executor}


\bibliographystyle{abbrv}
\bibliography{basic.bib}

\end{document}
