\documentclass{report}
\usepackage[utf8]{inputenc}
\usepackage{graphicx}
\usepackage{listings}
\usepackage{minted}
\usepackage{syntax}
\usepackage[acronym]{glossaries}
\usepackage[toc,page]{appendix}

\title{Training Basic Material}
\author{Marcelo Schmitt Laser}
\date{April 2017}

%-----------------------------------------------

\makeglossaries

\newacronym{bnf}{BNF}{Backus-Naur Form}
\newacronym{corba}{CORBA}{Common Object Request Broker Architecture}
\newacronym{cbse}{CBSE}{Component-Based Software Engineering}
\newacronym{spl}{SPL}{Software Product Line}
\newacronym{sple}{SPLE}{Software Product Line Engineering}
\newacronym{sei}{SEI}{Software Engineering Institute}
\newglossaryentry{bindtime}
{
	name={binding time},
	description={The moment during which the properties of a software artifact are configured. Typically, these may be: compile-time, where the configuration is handled during the compilation process; link-time, where the configuration is handled at the moment of dependency solution and access to external libraries; load-time, where the configuration is handled the moment the software product is instanced (loaded), and; run-time, where the configuration is handled during the software's execution}
}
\newglossaryentry{grammar}
{
	name={grammar},
	description={The set of rules that define the structure of all possible strings within a language. For any one string (regardless of size) to belong to a particular language, it must match that language's grammar. Unless stated otherwise, this document uses the terms \emph{grammar} and \emph{context-free grammar} interchangeably. More information on context-free grammars and grammars in general can be found in \cite{HOPCROFT:2000}}
}
\newglossaryentry{regex}
{
	name={regular expression},
	description={A sequence of characters that represents a string pattern. A regular expression has several uses, among which the ones we are most interested in are searching a stream of text for matching strings and validating a string against a pre-determined pattern. More information on regular expressions can be found in \cite{HOPCROFT:2000}}
}
\newglossaryentry{variability}
{
	name={variability},
	description={From SEI \cite{BACHMANN:2005}: ``Variability is the ability of a system, an asset, or a development environment to support the production of a set of artifacts that differ from each other in a preplanned fashion.''}
}
\newglossaryentry{variant}
{
	name={variant},
	description={From SEI \cite{BACHMANN:2005}: ``The realization of a variable part, meaning the result of exercising the variation mechanism(s), is called a variant.''}
}
\newglossaryentry{connector}
{
	name={connector},
	description={From Taylor et al. \cite{TAYLOR:2009}: ``An architectural element tasked with effecting and regulating interactions among components.''}
}
\newglossaryentry{vpoint}
{
	name={variation point},
	description={Also known as variable part, it is a well-defined point in the SPL where modifications may be introduced \cite{BACHMANN:2005}.}
}

%-----------------------------------------------

\begin{document}

%\maketitle

\begin{titlepage}
	\centering
	\par\vspace{3cm}
	{\scshape\LARGE Universidade Federal do Pampa \par}
	\vspace{4cm}
	{\huge\bfseries Training Basic Material\par}
	\vspace{2cm}
	{\Large\itshape Marcelo Schmitt Laser\par}

	\vfill

% Bottom of the page
	{\large \today\par}
\end{titlepage}

\chapter*{Introduction}

In order to properly implement the design decisions made for the PLeTs v4 project, it is necessary that all professional involved be aware of certain mechanisms, both theoretical and practical, that will be used in its design and development. This document is aimed to serve both as a basis for the study and explanation of these techniques and as a guide for the appropriate representation and implementation of them in PLeTs v4.

In Chapter \ref{ch:staticfinal} we will present the notion of using compilation directives for conditional compilation. First a basic theory of the concept is presented, based largely on the author's experience. The mechanism of using Static Final variables in Java is presented, along with its advantages and limitations. Then, the mechanisms of using Java annotations are evaluated. Finally, we analyse the need and properties of using an external preprocessor software, along with its perceived advantages.

In Chapter \ref{ch:patterns} we will present the basics of the software design patterns and architectural patterns used in PLeTs v4, with a focus on explaining their goals and implementation. First the Variant Factory design pattern will be presented \cite{LASER:2015} alongside the data structure catalogue system implemented in PLeTs v3. We then briefly expose the basic concepts of Software Connectors \cite{TAYLOR:2009}, with a particular emphasis on Linkage and Arbitrator connectors. With a basis on these concepts, we then proceed to present the architectural pattern proposed for the composition of products in PLeTs v4, which is largely based on the Distributed Objects architectural style (which, as far as we could find, was first used in \acrshort{corba} \cite{CORBA:2012}).

Chapter \ref{ch:modules} is meant to serve as a guide to the components designed within PLeTs v4 and their specifications. This chapter is focused on the architectural characteristics of these components, and is not meant to serve as a guide to the software testing methods applied in PLeTs.

Chapter \ref{ch:architecture} is reserved for track keeping of the evolution of the PLeTs v4 architecture project. It is my intention that this chapter be constantly updated and kept as a snapshot of the current state of the project. Any impreciseness or incorrectness in this chapter is to be taken as architectural erosion, and must be corrected as soon as possible.

Should this document at any point prove too dense, complicated, unclear or in any way lacking, I ask that you inform me (marcelo.laser@gmail.com) as quickly as possible so that the problem may be resolved and the document may be kept in the best possible shape. Please bear in mind that this document is to serve as the primary guide of the PLeTs v4 infrastructure, and therefore it aims to be as clear and precise as possible.

Marcelo Schmitt Laser


\printglossary[type=\acronymtype]
\printglossary

\tableofcontents

\chapter{Programming Paradigms}

%-----------------------------------------------

\section{Programming Paradigm vs Programming Language}

%-----------------------------------------------

\section{Programming Languages}

%-----------------------------------------------

\subsection{Basic Elements}

%-----------------------------------------------

\section{Paradigms}

%-----------------------------------------------

\subsection{Imperative}

%-----------------------------------------------

\subsubsection{Structured}

%-----------------------------------------------

\subsubsection{Procedural}

%-----------------------------------------------

\subsection{Object-Oriented}

%-----------------------------------------------

\subsection{Aspect-Oriented}

%-----------------------------------------------

\subsection{Declarative}

%-----------------------------------------------

\subsection{Functional}

%-----------------------------------------------

\subsection{Logic}

%-----------------------------------------------

\subsection{Event-Driven}

%-----------------------------------------------

\subsection{Reflective}

\chapter{Programming Logic}
\label{ch:logic}



%-----------------------------------------------

\section{Propositional Logic}

%-----------------------------------------------

\section{First-Order Logic}

%-----------------------------------------------

\section{Other Types}

\chapter{Modeling}

%-----------------------------------------------

\section{Fundaments}

%-----------------------------------------------

\subsection{Model vs Diagram}

%-----------------------------------------------

\subsection{View vs Visualization}

%-----------------------------------------------

\subsection{Structure vs Behaviour}

%-----------------------------------------------

\subsection{System vs Software}

%-----------------------------------------------

\subsection{Analysis vs Architecture}

%-----------------------------------------------

\section{Unified Modelling Language}

%-----------------------------------------------

\subsection{Fundamentals}

%-----------------------------------------------

\subsection{Use-Case Diagram}

%-----------------------------------------------

\subsection{Activity Diagram}

%-----------------------------------------------

\subsection{Class Diagram}

%-----------------------------------------------

\subsection{Component Diagram}

%-----------------------------------------------

\subsection{Sequence Diagram}

%-----------------------------------------------

\subsection{Communications Diagram}

%-----------------------------------------------

\subsection{State Machine Diagram}

%-----------------------------------------------

\subsection{Other UML Diagrams}

%-----------------------------------------------

\section{Other Models and Languages}

%-----------------------------------------------

\subsection{Entity-Relationship}

%-----------------------------------------------

\subsection{Flow Chart}

%-----------------------------------------------

\subsection{User Story}

%-----------------------------------------------

\subsection{Clafer}

%-----------------------------------------------

\subsection{XML}

%-----------------------------------------------

\subsection{xADL}

\chapter{Data Structures}
\label{ch:datastructures}

%-----------------------------------------------

\section{Fundamentals}

%-----------------------------------------------

\section{Array}

%-----------------------------------------------

\subsection{Matrix}

%-----------------------------------------------

\section{List}

%-----------------------------------------------

\subsection{Array List}

%-----------------------------------------------

\subsection{Linked List}

%-----------------------------------------------

\section{Stack}

%-----------------------------------------------

\section{Queue}

%-----------------------------------------------

\section{Tree}

%-----------------------------------------------

\subsection{Binary Tree}

%-----------------------------------------------

\subsection{Heap}

%-----------------------------------------------

\section{Graph}

%-----------------------------------------------

\subsection{Directed Graph}

%-----------------------------------------------

\subsection{Weighted Graph}

%-----------------------------------------------

\section{Class}

%-----------------------------------------------

\subsection{Object}

%-----------------------------------------------

\subsection{Struct}

%-----------------------------------------------

\section{Table}

%-----------------------------------------------

\subsection{Hash Table}

%-----------------------------------------------

\subsection{Dictionary}

%-----------------------------------------------

\subsubsection{Tuple}

\chapter{Algorithms}
\label{ch:algorithms}

%-----------------------------------------------

\section{Sorting Algorithms}

%-----------------------------------------------

\subsection{Insertion Sort}

%-----------------------------------------------

\subsection{Bubble Sort}

%-----------------------------------------------

\subsection{Quicksort}

%-----------------------------------------------

\subsection{Merge Sort}

%-----------------------------------------------

\section{Search Algorithms}

%-----------------------------------------------

\subsection{Breadth-First Search}

%-----------------------------------------------

\subsection{Depth-First Search}

%-----------------------------------------------

\subsection{A*}

%-----------------------------------------------

\subsection{Dijkstra's Algorithm}

%-----------------------------------------------

\subsection{Floyd-Warshall Algorithm}

\chapter{Patterns}

%-----------------------------------------------

\section{Fundamentals}

%-----------------------------------------------

\subsection{Design Pattern vs Architectural Pattern}

%-----------------------------------------------

\subsection{Architectural Pattern vs Architectural Style}

%-----------------------------------------------

\subsection{Architectural Style vs Programming Paradigm}

%-----------------------------------------------

\section{Architectural Styles}

%-----------------------------------------------

\subsection{Pipe-and-Filter}

%-----------------------------------------------

\subsection{Batch-Sequential}

%-----------------------------------------------

\subsection{Implicit Invocation}

%-----------------------------------------------

\subsection{Object-Oriented}

%-----------------------------------------------

\subsection{Distributed Objects (CORBA)}

%-----------------------------------------------

\subsection{Client-Server}

%-----------------------------------------------

\subsection{Mobile Code}

%-----------------------------------------------

\subsection{Peer-to-Peer}

%-----------------------------------------------

\subsection{Event-Based}

%-----------------------------------------------

\subsection{Publish-Subscribe}

%-----------------------------------------------

\subsection{Blackboard}

%-----------------------------------------------

\subsection{C2}

%-----------------------------------------------

\section{Architectural Patterns}

%-----------------------------------------------

\subsection{Model-View-Controller}

%-----------------------------------------------

\subsection{Broker}

%-----------------------------------------------

\subsection{Layers}

%-----------------------------------------------

\subsection{Sensor-Controller-Actuator}

%-----------------------------------------------

\subsection{Interpreter}

%-----------------------------------------------

\subsection{Application Controller}

%-----------------------------------------------

\section{Design Patterns}

%-----------------------------------------------

\subsection{Factory Method}

%-----------------------------------------------

\subsection{Abstract Factory}

%-----------------------------------------------

\subsection{Adapter}

%-----------------------------------------------

\subsection{Decorator}

%-----------------------------------------------

\subsection{Facade}

%-----------------------------------------------

\subsection{Observer}

%-----------------------------------------------

\subsection{Template Method}

%-----------------------------------------------

\subsection{Builder}

%-----------------------------------------------

\subsection{Prototype}

%-----------------------------------------------

\subsection{Singleton}

%-----------------------------------------------

\subsection{Iterator}

%-----------------------------------------------

\subsection{Mediator}

%-----------------------------------------------

\subsection{Memento}

%-----------------------------------------------

\subsection{State}

%-----------------------------------------------

\subsection{Strategy}

%-----------------------------------------------

\subsection{Visitor}

%-----------------------------------------------

\subsection{Data Transfer Object}

%-----------------------------------------------

\subsection{Plugin}

%-----------------------------------------------

\subsection{Separated Interface}

%-----------------------------------------------

\subsection{Component Configurator}

%-----------------------------------------------

\subsection{Monitor}

%-----------------------------------------------

\section{Connectors}

%-----------------------------------------------

\subsection{First-Class vs Second-Class}

%-----------------------------------------------

\subsection{Types of Connectors}

\chapter{Formal Languages}

%-----------------------------------------------

\section{Regular Expressions}

%-----------------------------------------------

\section{Automata}

%-----------------------------------------------

\subsection{Finite vs Infinite}

%-----------------------------------------------

\subsection{Deterministic vs Non-deterministic}

%-----------------------------------------------

\subsection{Probabilistic}

%-----------------------------------------------

\section{Lexical Analysis}

%-----------------------------------------------

\section{Backus-Naur Form}

%-----------------------------------------------

\section{Syntactic Analysis}

%-----------------------------------------------

\section{Semantic Analysis}

\chapter{Software Testing}
\label{ch:test}

%-----------------------------------------------

\section{Fundamentals}

%-----------------------------------------------

\subsection{Verification vs Validation}

%-----------------------------------------------

\subsection{Formal Verification vs Testing}

%-----------------------------------------------

\subsection{System Testing vs Software Testing}

%-----------------------------------------------

\section{Functional Testing}

%-----------------------------------------------

\section{Performance Testing}

%-----------------------------------------------

\section{Unit Testing}

%-----------------------------------------------

\section{Integration Testing}

%-----------------------------------------------

\section{Usability Testing}

%-----------------------------------------------

\section{Acceptance Testing}


\begin{appendices}

\end{appendices}

\bibliographystyle{abbrv}
\bibliography{basic.bib}

\end{document}
